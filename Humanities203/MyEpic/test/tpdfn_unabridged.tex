\documentclass[12pt]{article}

%
%Margin - 1 inch on all sides
%
\usepackage[letterpaper]{geometry}
\geometry{top=1.0in, bottom=1.0in, left=1.0in, right=1.0in}

%
%Doublespacing
%
\usepackage{setspace}
\doublespacing

% 
%Babel package for multiple language typesetting
%
%\usepackage[english,spanish]{babel}
%\usepackage[T1]{fontenc}
%\usepackage[latin1]{inputenc}

%
%Setting the font
%G
\usepackage{times}
\usepackage{graphicx}

\renewcommand{\thesection}{\arabic{section}.}

%
%Works cited environment
%(to start, use \begin{workscited...}, each entry preceded by \bibent)
% - from Ryan Alcock's MLA style file
%
\newcommand{\bibent}{\noindent \hangindent 40pt}
\newenvironment{workscited}{\newpage \begin{center} Works Cited \end{center}}{\newpage }

\newcommand{\mc}{M\'airt\'in}
\newcommand{\mcs}{M\'airt\'in }

%
%Begin document
%
\begin{document}
\begin{flushleft}

%%%%First page name, class, etc
Michael Gilliland \\
Dr. Linda Szabo \\
Humanities 203 \\
\today \\
Words: \\

%%%%Title
\begin{center}
{\large The Propitious Drones from Nevada}
\end{center}

%%%%Changes paragraph indentation to 0.5in
\setlength{\parindent}{0.5in} 

%START%

I chose the “Burn It Down scene” from Inglorious Basterds because it is a 
beautifully executed moment of transition from a state of surprise and 
confusion to one of massacre, chaos and explosion. This scene occurs 
shortly after he many gathered Nazis have been surprised by the face of 
Jewish Shosanna Dreyfus, hijacking the propaganda-ridden dialog of Nazi war 
hero, Fredrick Zoller. Through the reel, Shosanna then promises her theater's 
unwanted attendees, “You are all going to die.”

The first part of this clip is cut into from a brief glimpse of confused 
Goebbels who is denying that Shosanna's inflated face is a genuine 
component of his film. Then, from a viewpoint behind the screen, for 
the flash of a second, we see nothing but blue tinted, black and white 
lips saying, “Marcel.” Next, through the camera's eye, the viewers are 
transported a few yards back behind Marcel's right shoulder as he is 
looking intently at Shosanna's projected face.

In this cut in, Gobbels' propaganda is denied by the Jewish saboteur –-
his confusion is dismissed by the answering face of the new (soon to be) 
war hero, replacing that of Zoller (the older hero). The light blue 
saturated coloration of the old fashioned video looks ghostly, as if the 
recently deceased Shosanna is back from the grave, seeking revenge upon 
the Nazi killers who murdered her family and people. The cut in is of 
enormous importance because it provides a sense of hopelessness for the 
Nazis through the revelation of the new war hero's hidden motives being 
unveiled and her vengeance exacted.

At the beginning of the clip, behind the screen and Marcel's back we can
see only three things: Marcel, the top of the pile of nitrate film and 
Shosanna's ghostly face from the nose down. The lattermost of these 
three things is what our eyes are drawn to as we see and hear (the now 
slightly more distant lips) say, “Burn it down.” These three clearly 
annunciated words are essentially all we hear. There is some light murmuring 
of confused Nazis in the background, however, as far as the English-speaking 
viewer is concerned, their confusion should have been appeased by the phrase. 
Marcel, the projection of Shosanna and the film are the only objects we see 
because they are the only elements we care about at this moment. For the 
proper combination of these three alone will determine whether the 
protagonists are victorious.

Next the camera takes us in front of Marcel's face placed at an almost intimate 
distance in the dead center of the wide shot. We see him from his neckline 
to his hairline, the camera perpendicular to his face, looking from the 
direction of the screen but at a closer distance. To his left there are a few 
light wisps of smoke which are being emitted from the cigarette we later learn 
he is holding. He nods without ever blinking, accepting the symbolical torch 
that has been passed onto him by Shosanna. We see a black, void backdrop as 
we look into his smiling face. His honest smile is indicative of his deep 
love for Shosanna and his joy in seeing the successful concretion of their 
secret plan. His face it lit naturally and the little we see of his shirt 
is lit brightly, almost glowingly. The natural lighting emphasizes his 
humanity and loyalty in accepting his role in the massacre.

After Marcel's brief nod, he quickly mutters “Oui, Shosanna,” quickly, it is
almost difficult to understand what he is saying. What is interesting about 
this utterance is that it is unnecessary, almost redundant. The joyous 
emotion that the actor sets upon his face is enough to indicate his intention 
to fulfill the plot; the brief nod he provides further solidifies this 
intention; these words are almost overkill but not quite. By the time 
his words are finished the mumbling of the German theater patrons has dissolved.

We see a close-up of Marcel's hand planted on his side holding a smoking 
cigarette between his middle finger and thumb. The camera is at equal 
height with Marcel's side. As with the close-up of Marcel's face, the 
background is black, albeit there are some unrecognizable splotchy gray 
objects. This scene acts as the final “passing of the torch” scene. 
Marcel, in flicking the cigarette, is bringing his role in the 
destruction of the Nazis to an end and is leaving the rest up to the 
projectile cigarette. This brief scene's purpose is purely transitional. 
As he flicks the cigarette and it departs his hand at the center of the 
camera's view, a loud resonant flicking noise is heard and is held out. 
This is done to help ease the transition to the following scene filmed in 
slow-motion.

The cigarette floats, rolling over itself always almost perpendicularly to the line of sight of the camera. In this slow-motion shot there are three main physical planes that are visible: a plane in which the cigarette and smoke reside, a plane of moving specks of dust moving against the cigarette and finally the plane of the background. The cigarette and its trailing smoke spin throughout the flight of the cigarette, without significantly dropping as the camera tracks them. Although it is moving to the left its placement in the screen does not change significantly. The plane of specks, however, moves quickly against the cigarette. The specks seem to be the only thing aside from the actual flipping of the cigarette (and its slight drop) indicating its movement. In fact, the background and its shady, vaguely rectangular shapes drop below the cigarette creating a parallax effect as if the cigarette is truly flying.
                                    As it is airborne, the cigarette makes light scratching noises, as it struggles to work it way past the final obstacle that stands between it and the pile of film that will be downfall of the evil Nazi leaders – that obstacle being – the very air through which it soars.
                                        The flying cigarette scene is the longest in the sequence of the scenes present in the 15 second clip I selected and justly so. For this scene is the transition into the climax of the film. All the characters' plans have been made and many have died to see this moment through.
                                            Tarantino, however, does not let us get too attached to the cigarette. He abruptly ends the scene by jumping the camera to the nitrate film which bursts into red hot flame. The camera is focused on a small area of film and reels, looking down on it and showing the impact of the cigarette and the flame therefrom, but not much more. This close-up shot was used for two reasons: (1) the pile of film was unveiled earlier and showing the whole pile again would take away from the initial surprise and (2) showing more flame would take away from later burning scenes of a greater magnitude. The sound of the burning is a somewhat standard burning sound effect, however, there is a subtle buzzing noise mixed in which is presumably the nitrate being incinerated by the flame. This scene is very brief. At just about the time when you have received enough information about the burning film, the camera is transitioned to be above the audience, centering in on the theater screen to display one of the greatest scenes of the movie.
                                                In this excellent scene, we see the screen burning, the camera jumps to a height at level with the projected screen. Her face is dead center-screen, the proportions of her face to the rest of the view, interestingly, are about the same as Marcel's were in his close-up scene. Tarantino must like exposing the viewers to this personal sense of intimacy the last time we will see them in a film. We see, with red curtains on the sides, the red flame consuming Shosanna's gray-blue face as she is laughing maniacally.
                                                    This viewing angle of the screen has never been shown before. Earlier, when Shosanna's face appeared on the screen, we saw her, as one looking up, from the perspective of a person in the audience, as one of the Nazis. But now, in her victory, she is above the Nazis and she throws her head back and laughs as the flames consume the screen from the bottom up. She laughs, already having died and thus her projected form being rendered impervious to the flame – the Nazis are not so lucky.
                                                        As we are exposed to the burning screen, so are the Nazi attendees. The chaotic amalgamation of ladies' screams, the loudly burning flame and Shosanna's insane laughter growing more and more maniacal makes for a perfect accompaniment for the inferno of chaos being engendered and about to ensue. 
                                                            Yet another jumping camera transition is used to move into the next shot. This shot, truly unique to the short clip, is the only shot in which the camera appears to be moving. We see the defeat stricken Hitler, staring in disbelief, the camera zooming in on him and panning from about eye-level down to his chest in a radial manner (while still focusing on his expression). This shot is filmed in differently from the others to create a contrast. As said above, in the shot before this one we see Shosanna from a high point, laughing in victory. Well, in this shot, the camera sinks low, focusing in on Hitler's somber expression.
                                                                The actor who was selected for this movie to play Hitler is phenomenal. In most of the popular videos of WWII era Hitler we see him as being a very emotional individual, barking orders or giving an impassioned speech. In this shot, we see a new side to this emotional Hitler. His eyes are very wide, shocked in disbelief. Although this is a fairly short clip, we can see his mouth begin to open, his lips peeling slowly from one side to the other. He truly looks disturbed and surprised. 
                                                                    The final shot of this clip is transitioned into using, you guessed it, a camera jump. This time the camera is moved only about 5 feet back to show Goebbels' reaction (Goebbels is sitting next to Hitler). Goebbels and Hitler are both at the center of the screen (Goebbels on the left) and over Goebbels shoulder the face of his interpreter can be seen. Goebbels' reaction is every bit as distraught as Hitlers. We see his jaw drop, his eyebrows raise and his face form a hurtful expression of painful surprise.
                                                                        The director decided to transition from a shot of just Hitler's reaction to a shot of both him and Goebbels, for a somewhat obvious reason. 
                                                                            From these 9 shots and the rest of the theater scene an interesting plot irony should be recognized.  Shosanna's revenge is exacted in a parallel fashion as when, in the beginning of the film, her family was killed by BLANK's character, also know as The Jew Hunter. When The Jew Hunter comes to the house in which  Shosanna is hiding, he puts on a front, acting as though his visit has non-hostile purposes. When  Shosanna's family, hiding under the floor boards is not expecting it, the Jew hunter calls in his Nazi squad and they open fire on them (Shosanna barely escapes). Shosanna pretends to be sympathetic to the Nazis' aims, and she, like The Jew Hunter strikes unexpectedly.


In conclusion, it is clear that Tarantino enjoys constructing his plots around ironic justice. In the 
sixteen second clip in question, he utilizes steady cameras for all of the scenes except the one 
in which we see Hitler's reaction. During this heavily transitional clip, Tarantino seems to 
prefer jumping between shots in order to momentously continue through scenes while exposing the 
characters' reaction to their quickly changing surroundings. The modular use of blocks of 
color is also a predominant theme in this clip. A prime example of this would be the 
blue-gray figure of Shosanna's face being consumed by the growing orange flame. This 
clip marks the calm before the storm, as well as the first thunderclap. T, 
through the use of simple, not many objects usually placed at center-screen, 
delivers this transitional clip superbly – it is the culmination of the main characters' 
plans, marking the fictitious demise of the world's most malicient dictator.

%END%


%%%%Works cited
\begin{workscited}


\end{workscited}

\end{flushleft}
\end{document}
\}

